% !TEX TS-program = pdflatex
% !TEX encoding = UTF-8 Unicode

% This is a simple template for a LaTeX document using the "article" class.
% See "book", "report", "letter" for other types of document.

\documentclass[11pt]{article} % use larger type; default would be 10pt

\usepackage[utf8]{inputenc} % set input encoding (not needed with XeLaTeX)
\usepackage[english]{babel}

%%% Examples of Article customizations
% These packages are optional, depending whether you want the features they provide.
% See the LaTeX Companion or other references for full information.

%%% PAGE DIMENSIONS
\usepackage{geometry} % to change the page dimensions
\geometry{a4paper} % or letterpaper (US) or a5paper or....
% \geometry{margin=2in} % for example, change the margins to 2 inches all round
% \geometry{landscape} % set up the page for landscape
%   read geometry.pdf for detailed page layout information

\usepackage{graphicx} % support the \includegraphics command and options
\usepackage{wrapfig}

\usepackage[parfill]{parskip} % Activate to begin paragraphs with an empty line rather than an indent

%%% PACKAGES
\usepackage{booktabs} % for much better looking tables
\usepackage{array} % for better arrays (eg matrices) in maths
\usepackage{paralist} % very flexible & customisable lists (eg. enumerate/itemize, etc.)
\usepackage{verbatim} % adds environment for commenting out blocks of text & for better verbatim
\usepackage{subfig} % make it possible to include more than one captioned figure/table in a single float
% These packages are all incorporated in the memoir class to one degree or another...

%%% TILTLE AND AUTHOR

\title{
	Predictive Text
}


\author{Y3839090}
%\date{} % Activate to display a given date or no date (if empty),
         % otherwise the current date is printed 
\makeatletter

%%% HEADERS & FOOTERS
\usepackage{fancyhdr} % This should be set AFTER setting up the page geometry
\pagestyle{fancy} % options: empty , plain , fancy
\renewcommand{\headrulewidth}{1pt} % customise the layout...
\lhead{\@title}\chead{}\rhead{\@author}
\lfoot{}\cfoot{\thepage}\rfoot{}

%%% SECTION TITLE APPEARANCE
\usepackage{sectsty}
\allsectionsfont{\sffamily\mdseries\upshape} % (See the fntguide.pdf for font help)
% (This matches ConTeXt defaults)

%%% ToC (table of contents) APPEARANCE
\usepackage[nottoc,notlof,notlot]{tocbibind} % Put the bibliography in the ToC
\usepackage[titles,subfigure]{tocloft} % Alter the style of the Table of Contents
\renewcommand{\cftsecfont}{\rmfamily\mdseries\upshape}
\renewcommand{\cftsecpagefont}{\rmfamily\mdseries\upshape} % No bold!

% biblatex for citations
\usepackage{csquotes}
\usepackage[
backend=bibtex,
style=numeric,
sorting=ynt
]{biblatex}
\addbibresource{references.bib}
\usepackage{url}

\graphicspath{{Images/}}

%%% END Article customizations

%%% The "real" document content comes below...


\begin{document}

	\begin{titlepage}
		\centering
		\includegraphics[width=0.5\textwidth]{UoY_logo}\par\vspace{1cm}
		{\scshape\LARGE Department of Electronics \par}
		\vspace{1cm}
		{\scshape\Large Data Structures and Algorithms Assessment \par}
		\vspace{2cm}
		{\huge\bfseries Predictive Text\par}

		% maybe an image here
	
		\vfill
		{\Large\itshape \@author \par}
		\vspace{2cm}
		{\large \today\par}
	\end{titlepage}
	
	\tableofcontents
	\newpage
	
	\section{How the program works}
		dummy text
		\subsection{How users interact with the program}
			dummy text
		\subsection{How the prediction function works}
			dummy text

	\section{Data Structures}
		A predictive text program needs a to be able to access a set of common words.
		 Storing and accessing these words efficiently is one of the challenges in creating a fast predictive text engine.
		  I decided to use a trie\cite{book:ADS:trie} structure.
		  
		\subsection{Trie}
			\subsubsection{What is a trie}
			
				\begin{wrapfigure}{r}{0.33\textwidth} %this figure will be at the right
    				\centering
    				\includegraphics[width=0.33\textwidth]{Trie_example}
    				\caption{A trie for keys "A","to", "tea", "ted", "ten", "i", "in", and "inn"\cite{fig:Trie_example}.}
    				\label{fig:Trie_example}
				\end{wrapfigure}
				
				The trie is a tree-like data structure often used for storing strings. Each node in the trie represents a single letter in a string. Each node in a trie has a fixed number of child nodes like a binary search tree. Unlike the binary search tree the trie does not have two child nodes, it has one for each letter of the alphabet. This allows the trie nodes to not store their value because their position determines it. A node's children share a common prefix, that prefix is the value of there parent node.

There are many features of tries that make them a good choice for a predictive text system. One of them is that values can be look up by their prefixes. Predictive text systems use a partial word (a prefix) to search for possible full words. So it's important that prefix lookups are fast and efficient. [TODO:: ADD MORE STUFF HERE]
				
			\subsubsection{Time complexity of a trie}
				The search time for a value in a trie is \begin{math}O(len(string))\end{math}, where \begin{math}len(string)\end{math} is the length of the word to look up\cite{book:ADS:complexity}. This means that searches in a trie are linear with respect to the number of items in the trie. A predictive text function relies upon looks. Every time a user presses a key the system will preform at least one lookup and so it is vital that these searches be fast. 
				
				The time complexity of deletion in a trie is also linear. Deletions time complexity is 
				\begin{math}O(\vert A \vert len(string))\end{math}, where \begin{math}len(string)\end{math} is the length of the word to look up and \begin{math}\vert A \vert \end{math} is the size of the alphabet\cite{book:ADS:complexity}. In the use case as a Predictive text engine deletions will be very rare or non-existent. A simple implementation of predictive text may not allow the user to delete words form the dictionary at all. 
				
				The time complexity of insertion in a trie is the same as for deletions. Insertion time complexity is \begin{math}O(\vert A \vert len(string))\end{math}, where \begin{math}len(string)\end{math} is the length of the word to look up and \begin{math}\vert A \vert \end{math} is the size of the alphabet\cite{book:ADS:complexity}. When used in a predictive text engine most of the insertion into the trie will be made when loading the dictionary from file. 
			\subsubsection{Space complexity of a trie}
			
				
			\subsubsection{How I have implemented the trie}
		\subsection{Alternative Data structures}
			\subsubsection{Binary Trees vs Tries}
			
			\subsubsection{Hash map vs Tries}
			\subsubsection{List vs Tries}
	\section{Testing}
		\subsection{Complixity}
			\subsubsection{Time Complexity of Tries}
				

				
		\subsection{Unit Testing}
		\subsection{User interactions}
		
	\newpage
	\printbibliography
	\newpage
\end{document}
